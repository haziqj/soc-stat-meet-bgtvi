\subsection{The variational principle}

\begin{frame}{The variational principle}
  \vspace{-10pt}
  \begin{itemize}[<+->]
    \item Name derived from calculus of variations which deals with maximising or minimising functionals.
    \begin{table}
      \begin{tabular}{l  l  l }
      Functions   &$p:\theta \mapsto \bbR$  &(standard calculus) \\ 
      Functionals &$\cH:p \mapsto \bbR$     &(variational calculus) \\ 
      \end{tabular}
    \end{table}
  \item Using standard calculus, we can solve
  \[
    \argmax_\theta p(\theta) =: \hat\theta
  \]
  e.g. $p$ is a likelihood function, and $\hat\theta$ is the ML estimate.
  \item Using variational calculus, we can solve
  \[
    \argmax_p \cH(p) =: \tilde p
  \]
  e.g. $\cH$ is the entropy $\cH = - \int p(x)\log p(x) \d x$, and $\tilde p$ is the entropy maximising distribution.
  \end{itemize}
  \vspace{5pt}
\end{frame}

\subsection{Laplace's method}
\begin{frame}[label=laplace]{Laplace's method}
  \blfootnote{\fullcite[§4.1, pp.777-778.]{Kass1995}}
  \vspace{-15pt}
  \begin{itemize}[<+->]\setlength\itemsep{0.8em}
    \item Interested in $p(\bff|\by) \propto p(\by|\bff)p(\bff) =: e^{Q(\bff)}$, with normalising constant $p(\by) = \int e^{Q(\bff)} \d\bff$. The Taylor expansion of $Q$ about its mode $\tilde\bff$
    \[
      Q(\bff) \approx Q(\tilde\bff) - \half (\bff - \tilde\bff)^\top\bA(\bff - \tilde\bff) 
    \]
    is recognised as the logarithm of an unnormalised Gaussian density, with $\bA = -\text{D}^2 Q(\bff)$ being the negative Hessian of $Q$ evaluated at  $\tilde\bff$.
    \item The posterior $p(\bff|\by)$ is approximated by $\N(\tilde\bff, \bA^{-1})$, and the marginal by
    \[
      p(\by) \approx (2\pi)^{n/2} \vert \bA \vert^{-1/2}  p(\by|\tilde\bff)p(\tilde\bff)
    \]
    \item Won't scale with large $n$; difficult to find modes in high dimensions.
  \end{itemize}
  
  \begin{textblock*}{3cm}(.982\textwidth,1.04\textheight)%
    \hyperlink{estimation}{\beamerbutton{back}}      
  \end{textblock*}
\end{frame}

\begin{frame}[label=varcompare]{Comparison of approximations (density)}
  \vspace{-5pt}
  \only<1|handout:0>{
    \begin{center}
      \includegraphics[scale=0.7]{figure/compare1}
    \end{center}
  }
%  \only<2|handout:0>{
%    \begin{center}
%      \includegraphics[scale=0.7]{figure/compare2}
%    \end{center}
%  }  
  \only<2|handout:0>{
    \begin{center}
      \includegraphics[scale=0.7]{figure/compare3}
    \end{center}
  } 
%  \only<4|handout:0>{
%    \begin{center}
%      \includegraphics[scale=0.7]{figure/compare4}
%    \end{center}
%  } 
  \only<3|handout:1>{
    \begin{center}
      \includegraphics[scale=0.7]{figure/compare5}
    \end{center}
  } 
  
  \begin{textblock*}{3cm}(.982\textwidth,1.04\textheight)%
    \hyperlink{summary}{\beamerbutton{back}}      
  \end{textblock*}
\end{frame}

\begin{frame}{Comparison of approximations (deviance)}
  \vspace{-5pt}
  \begin{center}
    \includegraphics[scale=0.7]{figure/compare6}
  \end{center}

  \begin{textblock*}{3cm}(.982\textwidth,1.04\textheight)%
    \hyperlink{summary}{\beamerbutton{back}}      
  \end{textblock*}  
\end{frame}